\section{Conclusion} 
For the scope of this project, we explored different approaches to solving the Multiple Couriers Problem, each with its own modeling ideas. Working as a team, it was interesting to see how differently we tackled the problem and how our individual perspectives shaped the solutions.

All of the approaches produced almost equally optimal results for the first 10 instances, with only instance 7 from the MIP solution being suboptimal. However, as the problem size increased with more couriers and more items, the search space grew significantly, demanding more computational power and better strategies to navigate the complexity. In these larger instances, the CP approach delivered by far the best results, successfully solving all cases by effectively exploring the search space through Large Neighborhood Search. In contrast, the SMT approach was unable to find solutions for these instances, while the MIP approach solved only 2 out of the remaining 11, both with suboptimal results.

Thus, we can conclude that the CP approach is the most effective for solving the MCP problem. Beyond the results, this project was a valuable learning experience. It was our first encounter with this type of problem, and working through it gave us an engaging introduction to tackling NP-hard challenges in optimization.







